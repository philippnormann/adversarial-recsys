\chapter{Conclusion}
\label{chp:conclusion}
In this thesis, our goal was to investigate the vulnerability of a visual content-based \ac{RS} using a \ac{DNN} against adversarial attacks and to evaluate possible defense mechanisms. As a first step, we developed a new type of targeted item-to-item attack using state-of-the-art white-box methods and observed their effectiveness in compromising the integrity of the attacked \ac{RS}. In the next step, we tested two defense mechanisms utilizing \ac{AT} and were able to show that \ac{AT} had a significant positive impact on the robustness of our \ac{RS} against our performed attacks.

Although our experiments demonstrated a strong robustness of adversarially trained content-based \acp{RS} against our evaluated white-box attacks, it is unclear if and how far these results generalize for black-box or future unknown attacks.  Also, the effect of similar attacks and defenses on hybrid \acp{RS} using \acp{DNN} remains to be explored. Additionally, the trade-off in recommendation quality and robustness caused by \ac{AT} remains to be quantified, possibly by conducting user-surveys or A/B testing.

Overall, our findings have once again demonstrated the inherent vulnerability of \acp{DNN}, but 
have also given us hope that adversarially robust \ac{RS} models using \acp{DNN} might be within current reach.